%% This is an example first chapter.  You should put chapter/appendix that you
%% write into a separate file, and add a line \include{yourfilename} to
%% main.tex, where `yourfilename.tex' is the name of the chapter/appendix file.
%% You can process specific files by typing their names in at the 
%% \files=
%% prompt when you run the file main.tex through LaTeX.

\chapter{Introduction}
\section{What is Artisanal and Small-Scale Gold Mining}
Globally, about 100 million people depend on artisanal and small-scale gold mining (ASGM) for their livelihood, and artisanal and small-scale gold miners extract about 20\% of the world’ gold annually. Moreover, there are about 10 to 20 million (ASGM) miners worldwide, and they represent 90\% of the global gold mining workforce, and about a third of them are women \cite{planetgold_planetgold_2021}. ASGM is also an essential source of income and an opportunity for rural development. In the more than 70 countries where ASGM exists, there is a shortage of options and alternatives to ASGM for generating income to buy necessities of daily life. For example, in Peru, ASGM sustains the livelihoods of an estimated 1 million people, and between 300,000 and 500,000 miners were involved in Peru’s ASGM sector as of 2014. The ASGM definition varies in different countries but generally it is characterized by highly labor-intensive, low-capital-investment, essential tools, manual devices, or simple portable machinery methods of mining minerals. Furthermore, the Minamata Convention (MC) defines ASGM "as gold mining conducted by individuals or small enterprises with limited capital investment and production"\cite{unep_minamata_2019}.

Despite being a vital source of livelihood for the communities that practice ASGM, its activities often lead to several environmental, human, and social harms. ASGM externalities include mercury (Hg) releases to the environment, deforestation, tropical diseases such as malaria, dangerous and unsafe working conditions, crime and exploitation of indigenous communities, diesel and gasoline spills, and human trafficking. The reference of these various forms of impacts as externalities is relevant because in most cases the miners decide on their mining practices based only on the direct cost of and profit opportunity from production and sale of gold and do not consider the indirect costs to those harmed by the aforementioned impacts. Moreover, recent estimates assert that ASGM releases about 40\% of all Hg pollution to the environment, hence the largest Hg emissions source\cite{united_nations_environment_programme_technical_2019}. ASGM has become one of the focal areas targeted by measures under the Minamata Convention because it is estimated to be the highest source of anthropogenic Hg emissions to the atmosphere and releases to the hydrosphere. The emissions of Hg in ASGM are not only harmful to miners and members of their communities, but humans and ecosystems far away are also exposed to Hg risks because Hg travels globally through the atmosphere. Nevertheless, there is considerable uncertainty about the magnitude of ASGM Hg emissions and how much of the Hg used in ASGM travels globally.

\section{Minamata Convention}
Organic and inorganic forms of Mercury (Hg) in ecosystems have significant adverse effects on human health and the environment.The timeline of global efforts to better understand the human and environmental impacts of Hg date's back to the early 90s and the MC which was adopted in 2013 was the culmination of multiple years of scientific studies to show evidence that Hg had global effects \cite{unep_minamata_2019}. Moreover, the MC affirmed that global action is necessary to address the Hg pollution problem. The MC entered into force on 16 August 2017, on the $90^{th}$ day after the date of deposit of the $50^{th}$  instrument of ratification. The objective of the  MC is to protect human health and the environment from anthropogenic emissions and releases of mercury and mercury compounds\cite{united_nations_environment_programme_technical_2019}. 

\subsection{Measures to Address ASGM in the Minamata Concention}
ASGM is directly addressed in Article 7 and Annex C of the MC but, ASGM related issues such such as provisions on providing technical assistance and capacity building can also be tackled through other articles in line with executing the main objective of the MC. Article 7 requires countries where Hg is used in ASGM to develop a National Action Plan (NAP) that details ways to reduce and, where possible, eliminate the use of mercury and mercury compounds. Each country should include in its NAP actions to stop some of the worst practices of ASGM, which include, among other things, (i) whole ore amalgamation, (ii) open burning of amalgam, (iii) burning of amalgam in residential areas (iv), and cyanide leaching in sediment or tailings to which mercury has been added without first removing the mercury \cite{unep_minamata_2019}. Moreover, countries are required to include in their NAPs baseline estimates of the quantities of mercury used in ASGM and the produced estimate should have an accuracy of +/- 30\% and at worst +/- 50\%\cite{programme_estimating_2017}. Rather than using a generic strategy, countries must have a holistic approach to their NAPs, including policy, regulatory, institutional, technical, environmental, health, and socio-economic elements. Furthermore, international organizations have developed tools such as the planetGold Project (led by UNEP and Conservation International with specific programs in 23 developing countries) to facilitate countries' implementation of Article 7 of the MC. 

\subsection{Effectiveness Evaluation}
 A vital component of the MC is Article 22, which specifies a variety of information that must be included when conducting the effectiveness evaluation of the MC. The article states that "the Conference of the Parties shall,at its first meeting, initiate the establishment of arrangements for providing itself with comparable monitoring data on the presence and movement of mercury and mercury compounds in the environment as well as trends in levels of mercury and mercury compounds observed in biotic media and vulnerable populations."
 
\section{Organization and Thesis Questions}
Although consistent efforts have been made to address all the identified adverse effects of the ASGM sub-sector, few results have been produced. Addressing ASGM sub-sector issues requires finding ways to transform negative impacts into enhanced positive impacts, maximizing their contribution to poverty reduction, and creating resilient communities. Chapter 1 of this thesis provides general background on ASGM, its positive and negative impacts, and critical terminology concepts on modeling mercury in the atmosphere. Challenges of governments’ actions on ASGM monitoring are discussed, and examples of efforts at the national level are listed. 

Chapter 2 of the thesis starts with an extensive literature review of different Hg pollution estimation approaches. An overview of the main Hg pollution estimation methods is provided, and a list of case studies where ASGM Hg pollution is measured and associated results. Moreover, the application of a top-down technique to estimate Hg emissions by leveraging observation data from long-range monitoring stations and the GEOS chem model is presented.  

Chapter 3 presents the proposed conceptual framework for developing a network of Hg regional monitoring stations to improve the results of top-down Hg estimations and thus inform the effectiveness evaluation of the MC.

Top-down estimates of Hg in the atmosphere have a vital role in providing trusted, scientific-based, multi-scale (both spatial and temporal) and open data to decision-makers. Importantly, this requires that we transform raw data into information and knowledge to inform decisions, investments, consumers, and citizens. One challenge is to update this transformation process frequently because environmental issues (hence related decisions) may change rapidly. Another challenge is to include stakeholders in the design and implementation of the entire science-policy process. An inclusive approach is essential in the case of ASGM for building trust between the different stakeholders. 



%\section{Organization}