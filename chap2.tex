%% This is an example first chapter.  You should put chapter/appendix that you
%% write into a separate file, and add a line \include{yourfilename} to
%% main.tex, where `yourfilename.tex' is the name of the chapter/appendix file.
%% You can process specific files by typing their names in at the 
%% \files=
%% prompt when you run the file main.tex through LaTeX.
\chapter{Top Down Estimates of Mercury Emissions from ASGM Activities}
\section{Background}
\begin{flushleft}
Anthropogenic Hg sources, including ASGM, have been quantified in numerous prior studies using different methodologies. Bottom-up estimates leverage collected data on underlying activities and multiply activity levels by emission factors to estimate regional and global totals. For instance, the 2018 Global Mercury Assessment estimated ASGM Hg emissions to be 838 tonnes with an uncertainty range of 675-1000 tonnes for 2015\cite{united_nations_environment_programme_technical_2019}. Moreover, Streets et al. (2019) tested six different proxies for scaling emissions to other years and used an average value to scale the inventory of emissions to the year 2015 thus estimating that ASGM was the largest source and responsible for 775 Mg of emissions\cite{streets_global_2019}. Furthermore, Muntean et al. (2014) found that poverty in  gold ore rich countries (as measured by the GINI index \cite{sadefo_kamdem_nice_2012}, where available) was correlated with data on ASGM production activity when they used a poverty-based approach to estimate that ASGM was responsible for 728.27 t of emissions in 2010 which was equivalent to 41.1\% of the global emissions\cite{muntean_evaluating_2018}. Such inventories are comprehensible, and their universal applicability enables worldwide coverage. However, the different assumptions on the activity data and emission factors induce significant uncertainty in the emission inventories. An additional bias in the bottom-up approach originates from the reliance on officially reported emissions data, which might cause differences in accuracy between countries and regions.

National inventories are also a form of a bottom up inventory where countries identify and quantify the sources of mercury releases within their borders. Countries are required to include in their NAPs baseline estimates of the quantities of mercury used in ASGM and the produced estimate should have an accuracy of +/- 30\% and at worst +/- 50\% \cite{united_nations_environment_programme_estimating_2017}.
\end{flushleft}
\begin{flushleft}
On the contrary, top-down emission estimation approaches combine atmospheric transport and chemistry models with atmospheric concentration measurements to quantify emissions. Even though the atmospheric chemistry literature has various top-down method applications, no study explicitly constrains ASGM Hg emissions. For instance, Bousquet et al., 1999 applied top-down methods to infer surface fluxes of atmospheric CO\textsubscript{2} from observed concentrations\cite{bousquet_inverse_1999}. Furthermore, Kopacz et al., 2009, employed these techniques to quantify source contributions to ozone pollution at two adjacent sites on the U.S. west coast in the spring of 2006. Moreover, Stohl et al., 2010 determined the emissions of hydrochlorofluorocarbon and hydrofluorocarbon from four East Asian countries and the Taiwan region for the year 2008 using similar techniques. Hg emissions have also been constrained using top-down methods in Song et al., 2015 where they apply a top-down approach at a global scale to quantitatively estimate present-day Hg emission sources and critical parameters in GEOS-Chem to constrain the global biogeochemical cycle of Hg better. Moreover, Densler et al., 2017 used a top-down approach to quantify Hg emissions on a European scale based on the atmospheric Hg measurements conducted at the remote high altitude monitoring station, Jungfraujoch, Switzerland. 
\end{flushleft}
\begin{flushleft}
Herein we apply top-down techniques to investigate how ASGM Hg emissions can be estimated by leveraging simulations of Hg in the atmosphere generated by the GEOS-Chem model and observation data from long range Hg monitoring stations. By combining our top-down method with existing bottom-up data, we improve estimates of Hg emissions from ASGM activities, using Peru and the Madre de Dios region of South America as case studies.
\end{flushleft}

\section{Methods}
\subsection{Case Study Region}
\begin{flushleft}
As reported by the GMA 2018, Peru is one of the top emitters of ASGM Hg. Additionally, the Madre de Dios region was estimated to have released the largest quantities of Hg to the environment and the atmosphere[AGC]. This rainforest region lies between Bolivia and Brazil and covers roughly 85,000 square kilometers. The region's name is derived from the name of a major river that runs through it, and smaller streams and rivers cross through it to provide transportation and fishing for indigenous communities. In addition, these waterways are the main sites of ASGM and, subsequently, Hg contamination. 
\end{flushleft}
\begin{flushleft}
Although Brazil nut, coffee, papaya, and cacao are cultivated, the land is primarily dominated by ASGM (Diringer et al., 2015, Caballero Espejo et al., 2018). (Diringer et al., 2015) estimate that ASGM land has increased by 400\% since 1999. As a result, 100,000 hectares of forest have been deforested, with 10\% occurring only in 2017 (Caballero Espejo et al., 2018). Researchers have found it difficult to verify the amount of hg imported and gold extracted from Madre de Dios (Cortés-McPherson, 2019, Swenson et al., 2011, Veiga et al., 2006) and the number of illegal miners, which Mr. Brack-Egg estimated in 2010 at 30,000 (Diringer et al., 2015, MINAM 2011, Swenson et al., 2011). It is estimated that around 180 tons of Hg enter the environment of Madre de Dios every year (MongoBay Latam, 2018, Servindi, 2018). However, these numbers are not very accurate and are, in part, contradictory. 
\end{flushleft}
\begin{flushleft}
Hg has been extensively studied in Madre de Dios, both in people and in the environment. Prior environmental measurements performed in Madre de Dios regarding the amount of Hg entering soil, air, and waterways reflected drastically different results between government estimates and international researchers. This demonstrates a need to generate further system-wide constraints on ASGM Hg use and pollution. Some studies focus on human health in the region (Gibb \& O'Leary, 2014), and others have bridged natural science perspectives with community engagement and trans-disciplinary approaches (Landon et al., 2015). Through leveraging national and global emission inventories, long-range measurements of Hg concentration in the atmosphere, and the GEOS-Chem Chemical Transport Model, this project complements previous studies and distinguishes itself from previous studies in this highly-studied region. Results will be generated for the Madre de Dios region and its surrounding areas that have high ASGM activity. 
\end{flushleft}
\begin{figure}[h!]
  \includegraphics[scale=0.75]{templates/figures/Case_Study_region.png}
  \centering
  \caption{Map Showing the case study Region in Peru}
  \label{fig:PeruCS}
\end{figure}
\FloatBarrier

\begin{flushleft}
To develop a comprehensive understanding of how Hg pollution impacts the environment, long-term monitoring of ambient mercury data on a global scale is vital to assessing its emission, transportation, atmospheric chemistry, and deposition processes. In South America, especially in tropical regions, atmospheric mercury (Hg) data are rare; hence mercury dynamics are not well understood. The Global Mercury Observation System (GMOS) is one of only a few major global projects that aim to develop a standardized, coordinated global observing system for mercury pollution across the globe. Funded by the European Union, it comprises a vast network of ground-based monitoring stations, regular oceanographic cruises, and measurements in the lower and upper tropospheres and the lower stratosphere[cite]. More than 40 ground-based monitoring sites constitute the international network, covering many regions with limited to no observational data available before GMOS[cite]. Figure \ref{fig:GMOS_stations} shows the GMOS monitoring network sites in South America. Available Hg observation data from the GMOS stations on Figure \ref{fig:GMOS_stations} was obtained from the GMOS online database as well as published studies on the different sites. Atmospheric total gaseous mercury (TGM) data was  measured at these different stations using a Tekran analyzers (Tekran Inc., Toronto, Canada). 
\end{flushleft}

\begin{figure}[h]
  \includegraphics[scale=0.75]{templates/figures/GMOS_stations.png}
  \centering
  \caption{Map Showing the GMOS Monitoring Network Sites in South America}
  \label{fig:GMOS_stations}
  
\end{figure}
\FloatBarrier

\subsection{GEOS-Chem}
\begin{flushleft}
The GEOS-Chem model (Bey et al., 2001) is used to simulate atmospheric concentrations of Hg for comparison with data constraints. GEOS-Chem is a global-scale, 3-D atmospheric chemical transport model driven by meteorological input from the Goddard Earth Observing System (GEOS) of the NASA Global Modeling and Assimilation Office. It runs at a resolution of 0.25° latitude x 0.3125° longitude horizontally, equivalent to $\approx$27 km at the equator, and 72 levels in the vertical. Numerous atmospheric chemical species have been simulated using GEOS-Chem, including Hg (Selin et al., 2008, Zhang et al., 2016, Selin et al., 2007, Holmes et al., 2010, Amos et al., 2012). Travnikov et al. (2017) also used GEOS-Chem for international model comparisons to support the Global Mercury Assessment and the Minamata Convention on Mercury.
\end{flushleft}
\begin{flushleft}

Herein we simulated the Hg concentration in the atmosphere using version 12.8.0 of GEOS-Chem at a resolution of 2.0$\times$2.5, which is equivalent to a 200 km$\times$250 km grid square at the equator. Furthermore, the GMA 2018 emissions inventory was used to represent anthropogenic emissions sources from all sectors. GEOS-Chem enables researchers to toggle different emissions sources on or off depending on the research objective, hence a Baseline simulation was created by turning on all Hg emissions sources globally. Moreover, a No-ASGM simulation was generated by turning off the ASGM source globally to figure out the contribution of ASGM to the baseline Hg concentrations in the atmosphere,which is the difference between the No-ASGM Hg concentration and the Baseline Hg concentration in the atmosphere. GEOS-Chem also allows researchers to choose the frequency of the output Hg concentration hence the global simulation was set to output daily concentration of Hg in the atmosphere while the output for the grid boxes corresponding to the locations of the GMOS was set to a hourly frequency. The GEOS-Chem outputs for all the simulations conducted were in units of parts per trillion ($ppt$) and were converted to nano grams per meter cube ($ngm^{-3}$) to compare them to observations.
\end{flushleft}

\begin{flushleft}
The GEOS-Chem Hg model takes the emissions inventory as an input and output the Hg concentration in the atmosphere that results from the input emissions. Moreover, the Hg concentration output is directly proportional to the amount of input emissions. Consequently, the GMA 2018 ASGM emissions estimates from individual grid boxes in the case study region as seen in Figure \ref{fig:GMA_2018} were scaled and then used as input to the GEOS-Chem model. The emissions were scaled to investigate the relationship between the Hg concentration in the atmosphere and the changes in ASGM emissions from the case study region in Peru. Moreover, the Hg concentration signal resulting from scaling the emissions from a particular grid box was calculated using Equation \ref{doublingSig} below:
\end{flushleft}

\begin{flushleft}
\begin{equation}
\label{doublingSig}
Hg_{sig(region)}=\small\frac{(Hg_{m_1} -Hg_{m_0})}{(m_1 -m_0)}
\end{equation}
where:
\end{flushleft}

\begin{description}[leftmargin=!,labelwidth={3 em}]
    \item [$region$] is the specific department within the case study region.
    \item [$Hg_{m_1}$] is the Hg concentration signal at the observation site caused by scaling the emissions from a single grid box. 
    \item [$Hg_{m_0}$] is the baseline Hg concentration signal at the observation site when ASGM emissions are turned on in the GEOS-Chem simulation.
    \item [$m_1$] is the quantity of emissions in metric tonnes after scaling the emissions from a specific grid box.
\end{description}


\begin{flushleft}
$Hg_{sig(region)}$ gives the Hg concentration signal in the atmosphere that results from a unit change in the the tonnes of emissions from a specific grid box. Therefore, $Hg_{sig(region)}$ was used to investigate the sensitivity of the observations to the different amounts of additional ASGM Hg emissions and the regional grid boxes. The Hg concentration in the atmosphere that results from a specific change in emissions from a particular grid box was calculated using Equation \ref{ysignal} below.
\begin{equation}
\label{ysignal}
\small{Hg_{m(region)}} =Hg_{sig(region)}(m-m_o), 
\end{equation}
where:
\end{flushleft}


\begin{description}[leftmargin=!,labelwidth={1.5 em}]
    
    \item [$m_0$] is the GMA 2018 ASGM emissions estimate in metric tonnes for the particular grid box corresponding to a place in the case study region
    
    \item [$m$] is the amount of emissions, in metric tonnes, from a grid box required to produce $Hg_{m}$ concentration in the atmosphere
\end{description}

\begin{figure}[h]
  \includegraphics[scale=0.75]{templates/figures/GMA2018.png}
  \centering
  \caption{Map showing how the GMA2018 emission estimates for the year 2015 were distributed in Peru }
  \label{fig:GMA_2018}
  
\end{figure}
\FloatBarrier
\begin{flushleft}
For each grid box in the case study region, the emissions were modified from their original GMA 2018 estimates to new values based on Hg emission estimates produced by the Artisanal Gold Council(AGC)
\end{flushleft}
\subsection{Markov Chain Monte Carlo}

\begin{flushleft}
The Markov-Chain Monte Carlo (MCMC) is a sampling method that is also useful for fitting models to data\cite{hogg_data_2018}. We apply the MCMC to constrain ASGM Hg emissions from the case study region in Peru.The basic idea behind this approach is to compare the generated models to the data. The model is generated by a set of parameters, emissions, and we aim to sample from the set of parameters that best fits our data. The MCMC is used to compare the modelled concentrations to the observed data using metrics such as the $95^{th}$ confidence interval, mean and the interquartile range. The MCMC generatively models given data by sampling around optimum values from the posterior distribution. The MCMC is a Bayesian approach; hence it requires the definition of priors on the parameters of interest. The priors encode information that we already know of the system. The probability of the model given the observed data is given by the posterior probability, $P(\theta|D)$, which is calculated using the Bayes theorem:

\begin{equation}
\label{bayes_eq}
P(\theta|D)=\frac{P(D|\theta)P(\theta)}{P(D)}
\end{equation}
where:
\end{flushleft}

\begin{description}[leftmargin=!,labelwidth={3 em}]
    \item [$P(D|\theta)$] is the likelihood which is the probability of the data given the model
    \item [$P(\theta)$] is the prior which is the probability of the model and 
    \item [$P(D)$] is the evidence which is the probability of the data.
\end{description}

\begin{flushleft}
The MCMC enables the estimation of the sampling of the posterior distribution which is the left-hand side of Equation~\ref{bayes_eq}. To run the MCMC we establish a function that outputs a model given a set of input parameters. We modelled the Hg concentration at a particular grid box as a linear combination of Hg concentration signals from the case study region and the baseline Hg concentration as shown in the Equation \ref{Hg_conc} below:

\begin{align}
\begin{split}\label{Hg_conc}
Hg_{conc}= {}&Hg_{m(MdD)}+ Hg_{m(S-Puno)} + Hg_{m(N-Puno)} + Hg_{m(Apr)}+ Hg_{m(Aqp)}\\
            & +Hg_{m_0}
\end{split}
\end{align}

where:
\end{flushleft}

\begin{description}[leftmargin=!,labelwidth={5 em}]
    \item [$Hg_{m(MdD)}$] is the Hg concentration signal resulting from emissions from the Madre de Dios (MdD) grid box
    \item [$Hg_{m(S-Puno)}$] is the Hg concentration signal resulting from emissions from the South Puno (S-Puno) grid box
    \item [$Hg_{m(N-Puno)}$] is the Hg concentration signal resulting from emissions from the North Puno (N-Puno) grid box
    \item [$Hg_{m(Apr)}$] is the Hg concentration signal resulting from emissions from the Apurimac (Apr) grid box
    \item [$Hg_{m(Aqp)}$] is the Hg concentration signal resulting from emissions from the Arequipa (Aqp) grid box
    \item [$Hg_{m_0}$] is the baseline Hg concentration signal.
\end{description}

\begin{flushleft}
Each of the $Hg_{m(region)}$ terms of Equation \ref{Hg_conc} represent signals from the different departments are calculated using Equation~\ref{ysignal}. Equation~\ref{ysignal} The $m_(region)$ terms are the only unknowns in  and the equation can be expanded to isolate the terms with $m_(region)$, which is the parameter we are optimizing for in the MCMC method. The expanded form of Equation \ref{Hg_conc} is shown below:

\begin{align}
\begin{split}\label{Cs36PoGd2l}
Hg_{conc}={}& (m_{(MdD)}Hg_{sig_{(MdD)}} -m_oHg_{sig_{(MdD)}})+ (m_{(S-Puno)}Hg_{sig_{(S-Puno)}} -m_oHg_{sig_{(S-Puno)}}) \\
            &+ (m_{(N-Puno)}Hg_{sig_{(N-Puno)}} -m_0Hg_{sig_{(N-Puno)}}) + (m_{(Apr)}Hg_{sig_{(Apr)}} -m_oHg_{sig_{(Apr)}}) \\
            &+ (m_{(Aqp)}Hg_{sig_{(Aqp)}} -m_oHg_{sig_{(Aqp)}})+Hg_{m_0}
\end{split}
\end{align}

Since the values of $m_{(region)}$ are the parameters that we want to estimate using MCMC, they can represented as $\theta_i=m_{(region)}, i=1$ and the other terms including the background concentration are combined into one constant, C:

\begin{equation}
\begin{aligned}
    Hg_{conc}  & = \theta_0C  + \theta_1Hg_{sig_{(MdD)}}+ \theta_2Hg_{sig_{(S-Puno)}} +  \theta_3Hg_{sig_{(N-Puno)}} \\
                & \ \ \ \  +\theta_4Hg_{sig_{(Apr)}} +  \theta_5Hg_{sig_{(Aqp)}}
\end{aligned}
\end{equation}

\begin{align}
Hg_{conc} =\begin{bmatrix} C & Hg_{sig_{(MdD)}} & Hg_{sig_{(S-Puno)}} &Hg_{sig_{(N-Puno)}} &Hg_{sig_{(Apr)}} &Hg_{sig_{(Aqp)}}\end{bmatrix} \times 
            \begin{bmatrix} \theta_0 \\ \theta_1 \\ \theta_2\\ \theta_3\\ \theta_4\\ \theta_5  \end{bmatrix}
\end{align}
where $\theta_0=1$ and $Hg_{conc}$ is the modeled Hg concentration at the observation site of interest.
\end{flushleft}

\section{Results and Discussion}
\begin{flushleft}


Initial results, as seen in Figure \ref{fig:GMOSplots} indicated that the ASGM emissions related Hg concentration signal created by the GEOS-Chem model is prominent in only one of the GMOS sites. This is mountain site at the Chalcataya (CHC) regional station, Figure \ref{fig:GMOSplots} (a), (World Meteorological Organization, WMO, region III-- South America; 16.35023◦ S, 68.13143◦ W),which is at an altitude of 5240 meters above sea level, about 140m below the summit of mount Chacaltaya on the eastern edge of the ``Cordillera Real,'' with a horizon open to the south and west. CHC is about 300km from the ``Madre de Dios'' watershed. Further analysis and comparisons were carried out based on the data from the CHC site. Moreover, using the CHC site was pertinent because it is also the closest to the case study region. 
\end{flushleft}
\begin{figure}[h]

\begin{tabular}[h]{cc}

\subfloat[Chalcataya]{\includegraphics[width = 0.45\linewidth]{templates/figures/CHC.png}} &
\subfloat[Bariloche]{\includegraphics[width = 0.45\linewidth]{templates/figures/BAR.png}}\\

\subfloat[Calhau]{\includegraphics[width = 0.45\linewidth]{templates/figures/CAL.png}} &
\subfloat[Niew Nickerie]{\includegraphics[width = 0.45\linewidth]{templates/figures/NIK.png}}\\

\subfloat[Manaus]{\includegraphics[width = 0.45\linewidth]{templates/figures/MAN.png}} &
\subfloat[Sisal]{\includegraphics[width = 0.45\linewidth]{templates/figures/SIS.png}}\\

\end{tabular}
\centering
\captionof{figure}{Plots Showing Data and Modeling  Results from GMOS Monitoring Network Sites in South America}
\label{fig:GMOSplots}
\end{figure}
\FloatBarrier

\begin{flushleft}
Figure\ref{fig:ModelvsObs} shows the comparison of the GEOS-Chem Model time series output for a high resolution simulation (green) and low resolution simulation (blue) with observations (red) from CHC over the period from July 2014 to January 2016. review of the plot reveals that the observations follow a trend that is not observed in the model. Furthermore, increasing the resolution of the model increases the peaks in the GEOS-chem output Hg concentration. Koening et all, assert that the upward trend in the Hg concentration in the observations is due to ENSO hence they categorize the observation data into normal conditions (NC), 2014-07 to 2015-05 and ENSO conditions  2015-06 to 2016-01. Since the GEOS chem Hg model assumes normal conditions when creating the simulation, the data that was recorded under normal conditions was used in the analysis that follows. Another critical observation over the NC period is the fact that the model overestimated the Hg Concentration in the atmosphere.   
\end{flushleft}


\begin{figure}[h]
  \includegraphics[scale=0.5]{templates/figures/ModelvsObs/ModelvsObs_v1.png}
  \centering
  \caption{Comparison of the GEOS-Chem Model time series output for a high resolution simulation and low resolution simulation with observations from CHC over the period from July 2014 to January 2016}
  \label{fig:ModelvsObs}
\end{figure}
\FloatBarrier

\begin{flushleft}
Furthermore, a comparison of the modelled and observed  Hg concentration at CHC using a probability density plots as seen in Figure \ref{fig:Histplots} showed that the GEOS-Chem model recreates the interquatile range of the observations better that it recreates the mean Hg concentration in the atmosphere when ASGM Hg emissions are turned on. Figure \ref{fig:Histplots}(c) shows that the variability in the Hg concentration at CHC is influenced by ASGM Hg emissions.
\end{flushleft}

\begin{figure}[h]

\begin{tabular}[h]{cc}

\subfloat[]{\includegraphics[width = 0.45\linewidth]{templates/figures/ModelvsObs/GC_noASGMvsObs.png}} &
\subfloat[]{\includegraphics[width = 0.45\linewidth]{templates/figures/ModelvsObs/GCasgm_vsObs.png}}\\

\subfloat[]{\includegraphics[width = 0.45\linewidth]{templates/figures/ModelvsObs/Hist_GCbase_vs_GCasgm_vs_obs_v1.png}} &
\subfloat[]{\includegraphics[width = 0.45\linewidth]{templates/figures/ModelvsObs/GCasgm_vsObs_standardized.png}}
\end{tabular}
\centering
\captionof{figure}{Plots Showing Data and Modeling  Results from GMOS Monitoring Network Sites in South America}
\label{fig:Histplots}
\end{figure}
\FloatBarrier


\begin{comment}

\begin{figure}[h]
  \includegraphics[scale=0.5]{templates/figures/ModelvsObs/GC_noASGMvsObs.png}
  \centering
  \caption{Histogram of Hg concentration for model output when the ASGM Hg source is turned off and the observations for the normal period}
  \label{fig:GMOS_stations}
\end{figure}
\FloatBarrier

\begin{figure}[h]
  \includegraphics[scale=0.5]{templates/figures/ModelvsObs/GCasgm_vsObs.png}
  \centering
  \caption{Histogram of Hg concentration for model output when the ASGM Hg source is turned on and the observations for the normal period}
  \label{fig:GMOS_stations}
\end{figure}
\FloatBarrier

\begin{figure}[h]
  \includegraphics[scale=0.5]{templates/figures/ModelvsObs/Hist_GCbase_vs_GCasgm_vs_obs_v1.png}
  \centering
  \caption{Histogram of Hg concentration for model output when the ASGM Hg source is turned off (green),when its turn on (blue)and the observations (red) for the normal period}
  \label{fig:GMOS_stations}
\end{figure}
\FloatBarrier

\begin{figure}[h]
  \includegraphics[scale=0.5]{templates/figures/ModelvsObs/GCasgm_vsObs_standardized.png}
  \centering
  \caption{Histogram of standardized Hg concentration for model output when the ASGM Hg source is turned on (blue) and the observations (red) for the normal period}
  \label{fig:GMOS_stations}
\end{figure}
\FloatBarrier
\end{comment}
\begin{flushleft}

\end{flushleft}
\begin{figure}[h]
  \includegraphics[scale=0.5]{templates/figures/ModelvsObs/CorrCompare.png}
  \centering
  \caption{Comparison of correlations between model output and observations}
  \label{fig:GMOS_stations}
\end{figure}
\FloatBarrier

\section{Policy Implications}

