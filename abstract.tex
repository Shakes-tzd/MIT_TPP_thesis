% $Log: abstract.tex,v $
% Revision 1.1  93/05/14  14:56:25  starflt
% Initial revision
% 
% Revision 1.1  90/05/04  10:41:01  lwvanels
% Initial revision
% 
%
%% The text of your abstract and nothing else (other than comments) goes here.
%% It will be single-spaced and the rest of the text that is supposed to go on
%% the abstract page will be generated by the abstractpage environment.  This
%% file should be \input (not \include 'd) from cover.tex.
A vital aspect of the Minamata Convention is reducing and eventually eliminating mercury emissions from artisanal and small-scale gold mining (ASGM). ASGM is the world's largest source of anthropogenic Hg emissions and is common in Latin America, Sub-Saharan Africa, South Asia, and East Asia. However, the amount of mercury emitted from ASGM and contributing to global mercury emissions is subject to substantial uncertainty. Sources of Hg, including ASGM, have been quantified in bottom-up studies that collect data on underlying activities and multiply them by emission factors to estimate regional and global totals. In contrast, top-down studies have quantitatively used atmospheric concentration measurements and models to constrain Hg emissions. However, no top-down estimates have yet been compiled for ASGM emissions. With GEOS-Chem's global-scale chemical transport model for Hg, we investigate whether and how ASGM-related Hg emissions can be quantified using existing and potential regional measurement sites for gaseous elemental mercury (GEM). By combining our top-down method with existing bottom-up data, we improve estimates of Hg emissions from ASGM activities, using Peru and the Madre de Dios region of South America as case studies. In the course of our model-based analysis and comparison with existing observations, we find that information on the shape of the probability distribution of GEM concentrations (including the interquartile range and 95\% range) provides better quantitative constraints on ASGM emissions than long-term mean values, suggesting possible design guidelines for monitoring networks. Our model-based analysis offers insights into improving ASGM emissions estimates at the regional scale.