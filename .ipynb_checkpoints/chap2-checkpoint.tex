%% This is an example first chapter.  You should put chapter/appendix that you
%% write into a separate file, and add a line \include{yourfilename} to
%% main.tex, where `yourfilename.tex' is the name of the chapter/appendix file.
%% You can process specific files by typing their names in at the 
%% \files=
%% prompt when you run the file main.tex through LaTeX.
\chapter{Top Down Estimates of Mercury Emissions from ASGM Activities}
\section{Background}
Anthropogenic Hg sources, including ASGM, have been quantified in numerous prior studies using different methodologies. Bottom-up estimates leverage collected data on underlying activities and multiply activity levels by emission factors to estimate regional and global totals. For instance, the 2018 Global Mercury Assessment estimated ASGM Hg emissions to be 838 tonnes with an uncertainty range of 675-1000 tonnes for 2015 (AMAP 2019). Moreover, Streets et al. (2019) tested six different proxies for scaling emissions to other years and used an average value to scale the inventory of emissions to the year 2015 thus estimating that ASGM was the largest source and responsible for 775 Mg of emissions. Furthermore, Muntean et al. (2014) found that poverty in  gold ore rich countries (as measured by the GINI index (Kamdem, 2012), where available) was correlated with data on ASGM production activity when they used a poverty-based approach to estimate that ASGM was responsible for 728.27 t of emissions in 2010 which was equivalent to 41.1\% of the global emissions . Such inventories are comprehensible, and their universal applicability enables worldwide coverage. However, the different assumptions on the activity data and emission factors induce significant uncertainty in the emission inventories. An additional bias in the bottom-up approach originates from the reliance on officially reported emissions data, which might cause differences in accuracy between countries and regions.

National inventories are also a form of a bottom up inventory where countries identify and quantify the sources of mercury releases within their borders. Countries are required to include in their NAPs baseline estimates of the quantities of mercury used in ASGM and the produced estimate should have an accuracy of +/- 30\% and at worst +/- 50\% (UNEP, 2017)

On the contrary, top-down emission estimation approaches combine atmospheric transport and chemistry models with atmospheric concentration measurements to quantify emissions. Even though the atmospheric chemistry literature has various top-down method applications, no study explicitly constrains ASGM Hg emissions. For instance, Bousquet et al., 1999 applied top-down methods to infer surface fluxes of atmospheric CO\textsubscript{2} from observed concentrations. Furthermore, Kopacz et al., 2009, employed these techniques to quantify source contributions to ozone pollution at two adjacent sites on the U.S. west coast in the spring of 2006. Moreover, Stohl et al., 2010 determined the emissions of hydrochlorofluorocarbon and hydrofluorocarbon from four East Asian countries and the Taiwan region for the year 2008 using similar techniques. Hg emissions have also been constrained using top-down methods in Song et al., 2015 where they apply a top-down approach at a global scale to quantitatively estimate present-day Hg emission sources and critical parameters in GEOS-Chem to constrain the global biogeochemical cycle of Hg better. Moreover, Densler et al., 2017 used a top-down approach to quantify Hg emissions on a European scale based on the atmospheric Hg measurements conducted at the remote high altitude monitoring station, Jungfraujoch, Switzerland.

The purpose of this thesis project is to apply top-down techniques to investigate how ASGM Hg emissions can be estimated by leveraging simulations of Hg in the atmosphere generated by the GEOS-Chem model and observation data from long range Hg monitoring stations. 

Similarly, this study leverages atmospheric Hg measurements conducted at the remote high altitude monitoring station in Chalcataya, Bolivia, and the GEOS-Chem model to estimate ASGM emissions from four regions in Peru known to have high ASGM activities.For atmospheric emissions, gaseous elemental mercury (GEM), gaseous oxidized mercury (GOM), and particle-bound mercury (PBM) are the Hg forms of concern. GEM is the most common form of mercury in the atmosphere because of its high vapor pressure and long residence time, enabling it to be transported over long distances and spread globally.


\section{Methods}


\subsection*{GEOS-Chem}

Better estimates of emissions are obtained by using the GEOS-Chem model (Bey et al., 2001) to simulate atmospheric concentrations of Hg for comparison with data constraints. GEOS-Chem is a global-scale, 3-D atmospheric chemical transport model driven by meteorological input from the Goddard Earth Observing System (GEOS) of the NASA Global Modeling and Assimilation Office. It runs at a resolution of 0.25° latitude x 0.3125° longitude horizontally, equivalent to $\approx$27 km at the equator, and 72 levels in the vertical. A variety of atmospheric chemical species of interest, including Hg (Selin et al., 2008, Zhang et al., 2016, Selin et al., 2007, Holmes et al., 2010, Amos et al., 2012) have been simulated on GEOS-Chem. Moreover, GEOS-Chem has also been used in international model comparisons to support the Global Mercury Assessment and the Minamata Convention on Mercury (Travnikov et al., 2017).


\subsection*{Measurement Site and Data}

The measurements were conducted at the CHC GAW regional station (World Meteorological Organization, WMO, region III-- South America; 16.35023◦ S, 68.13143◦ W), at an altitude of 5240ma.s.l., about 140m below the summit of mount Chacaltaya on the eastern edge of the ``Cordillera Real,'' with a horizon open to the south and west. The station's area is rocky, sparsely vegetated, and has intermittent snow cover (especially in the wet season). CHC is about 300km from the ``Madre de Dios'' watershed, a known ASGM hot spot (Beal et al., 2013; Diringer et al., 2015, 2019) and one of the areas whose emissions are estimated in this study. In addition to the Madre de Dios region, many other ASGM sites exist in the Bolivian, Peruvian, and Brazilian lowlands. Atmospheric total gaseous mercury (TGM) was measured at the CHC GAW station from July 2014 to February 2016 using a Tekran Model 2537A analyzer (Tekran Inc., Toronto, Canada). Concentrations are expressed in nanograms per cubic meter at standard temperature and pressure (STP; 273.15K, 1013.25hPa).

\subsection*{Markov Chain Monte Carlo}

The Markov-Chain Monte Carlo (MCMC) is a sampling method that is also useful for fitting models to data. This paper applies the MCMC to constrain ASGM Hg emissions from five regions in Peru by comparing the model of the Hg concentration in the atmosphere against observed Hg concentrations using metrics such as the $95^{th}$ confidence interval and the interquartile range. MCMC generatively models given data by sampling around optimum values from the posterior distribution. The model is generated by a set of parameters that include the emissions from different grid boxes in Peru. The MCMC is a Bayesian approach; hence it requires the definition of priors on the parameters of interest. The priors encode information that we already know of the system. The probability of the model given the observed data is given by the posterior probability, $P(\theta|D)$, which is calculated using the Bayes theorem:



\begin{equation}
\label{SLo2IFBPIH}
P(\theta|D)=\frac{P(D|\theta)P(\theta)}{P(D)}
\end{equation}



where $P(D|\theta)$ is the likelihood which is the probability of the data given the model, $P(\theta)$ is the prior which is the probability of the model and P(D) is the evidence which is the probability of the data.

The MCMC enables the estimation of the sampling of the posterior distribution which is the left-hand side of Equation~\ref{SLo2IFBPIH}. To run the MCMC we establish a function that outputs a model given a set of input parameters.



\begin{equation}
\label{zjLacpI3KN}
\small{y_{signal}} =\small\frac{(x_{modified} -x_{base})}{(m_1 -m_0)}\small{(m -m_o)}
\end{equation}

\begin{itemize}[labelwidth={3.5em},font=\bfseries,align=left]
  \item [$m_0$] GMA 2018 emissions estimate in metric tonnes for the particular grid box
  \item [$m_1$] $2\times m_0$
  \item [$m$] Amount of emissions from a grid box required to produce $y_{signal}$
  \item [$y_{signal}$] Modeled atmospheric Hg concentration signal when the emissions from a particular grid box are equal to $m$ tonnes
  \item [$x_{modified}$] Modeled atmospheric Hg concentration when the emissions from a particular grid box are equal to $m_1$
  \item [$x_{base}$] Modeled Hg concentration when the emissions from a grid box are equivalent to the emissions in the inventory used to run the simulation
\end{itemize}
Considering that $m$ is the only unknown variable in Equation~\ref{zjLacpI3KN}, the equation can be expanded to separate $m$ which is the parameter we are optimizing for in the MCMC method:

\begin{equation}
\label{jzqAiCqIzU}
\small{y_{signal}} =\small\frac{(x_{modified} -x_{base})}{(m_1 -m_0)}m -\small\frac{(x_{modified} -x_{base})}{(m_1 -m_0)}m_o
\end{equation}



The equation can be simplified by the following:

\begin{equation}
\label{LjJwnQMt2P}
w=\small\frac{(x_{modified} -x_{base})}{(m_1 -m_0)}
\end{equation}



The modeled atmospheric Hg concentration at CHC is then represented as a linear combination of the background concentration and the Hg concentration signals from the grid boxes around the site. We assume that grid boxes in Peru are responsible for the observed ASGM signal at CHC which is represented by the equation below:

\begin{align}
\begin{split}\label{oqWtdKQu1C}
Hg_{(CHC)}= {}&Hg_{(MdD )}+ Hg_{(S -Puno)} + Hg_{(N -Puno)} + Hg_{(Apr)}+ Hg_{(Aqp)}\\
            & +Hg_{background}
\end{split}
\end{align}

where each $HgSignal$ is given by $y_{signal}$ from Equation~\ref{jzqAiCqIzU} above hence can be simplified in to:



\begin{align}
\begin{split}\label{Cs36PoGd2l}
Hg_{(CHC)}={}& (w_{(MdD)}m -w_{(MdD)}m_o)+ (w_{(S-Puno)}m -w_{(S-Puno)}m_o) \\
            &+ (w_{(N-Puno)}m -w_{(N-Puno)}m_o) + (w_{(Apr)}m -w_{(Apr)}m_o) \\
            &+ (w_{(Aqp)}m -w_{(Aqp)}m_o)+Hg_{background}
\end{split}
\end{align}



$m$ is the parameter that we want to estimate using MCMC $\theta=m$ and the other terms including the background concentration are combined into one constant, C:




\begin{equation}
\begin{aligned}
    Hg_{(CHC)}  & = C \theta_0 + w_{(MdD)} \theta_1+ w_{(S-Puno)} \theta_2 + w_{(N-Puno)} \theta_3 \\
                & + w_{(Apr)} \theta_4+  w_{(Aqp)} \theta_5
\end{aligned}
\end{equation}



\begin{align}
y_{model}=\begin{bmatrix} C & w_{(MdD)} & w_{(S-Puno)} &w_{(N-Puno)} &w_{(Apr)} &w_{(Aqp)}\end{bmatrix} × \begin{bmatrix} \theta_0 \\ \theta_1 \\ \theta_2\\ \theta_3\\ \theta_4\\ \theta_5 \end{bmatrix}
\end{align}

%% https://curvenote.com/oxa:J4aS4MEiOL7h1cpaHIvX/iADqtVfkyzlbAkh3xFEC.1

where $\theta_0=1$ and $y_{model}$ is the modeled Hg concentration at Chalcataya.


\section{Results and Discussion}

\subsection{Another subsection sample}

\section{Policy Implications}

