% -*-latex-*-
% 
% For questions, comments, concerns or complaints:
% thesis@mit.edu
% 
%
% $Log: cover.tex,v $
% Revision 1.9  2019/08/06 14:18:15  cmalin
% Replaced sample content with non-specific text.
%
% Revision 1.8  2008/05/13 15:02:15  jdreed
% Degree month is June, not May.  Added note about prevdegrees.
% Arthur Smith's title updated
%
% Revision 1.7  2001/02/08 18:53:16  boojum
% changed some \newpages to \cleardoublepages
%
% Revision 1.6  1999/10/21 14:49:31  boojum
% changed comment referring to documentstyle
%
% Revision 1.5  1999/10/21 14:39:04  boojum
% *** empty log message ***
%
% Revision 1.4  1997/04/18  17:54:10  othomas
% added page numbers on abstract and cover, and made 1 abstract
% page the default rather than 2.  (anne hunter tells me this
% is the new institute standard.)
%
% Revision 1.4  1997/04/18  17:54:10  othomas
% added page numbers on abstract and cover, and made 1 abstract
% page the default rather than 2.  (anne hunter tells me this
% is the new institute standard.)
%
% Revision 1.3  93/05/17  17:06:29  starflt
% Added acknowledgements section (suggested by tompalka)
% 
% Revision 1.2  92/04/22  13:13:13  epeisach
% Fixes for 1991 course 6 requirements
% Phrase "and to grant others the right to do so" has been added to 
% permission clause
% Second copy of abstract is not counted as separate pages so numbering works
% out
% 
% Revision 1.1  92/04/22  13:08:20  epeisach

% NOTE:
% These templates make an effort to conform to the MIT Thesis specifications,
% however the specifications can change. We recommend that you verify the
% layout of your title page with your thesis advisor and/or the MIT 
% Libraries before printing your final copy.
\title{Assessing the role of top-down techniques for improving regional estimates of artisanal and small-scale gold mining mercury emissions}

\author{Thandolwethu Dlamini}
% If you wish to list your previous degrees on the cover page, use the 
% previous degrees command:
%       \prevdegrees{A.A., Harvard University (1985)}
% You can use the \\ command to list multiple previous degrees
%       \prevdegrees{B.S., University of California (1978) \\
%                    S.M., Massachusetts Institute of Technology (1981)}
\department{Technology and Policy Program}

% If the thesis is for two degrees simultaneously, list them both
% separated by \and like this:
% \degree{Doctor of Philosophy \and Master of Science}
\degree{Master of Science in Technology and Policy}

% As of the 2007-08 academic year, valid degree months are September, 
% February, or June.  The default is June.
\degreemonth{September}
\degreeyear{2022}
\thesisdate{August 05, 2022}

%% By default, the thesis will be copyrighted to MIT.  If you need to copyright
%% the thesis to yourself, just specify the `vi' documentclass option.  If for
%% some reason you want to exactly specify the copyright notice text, you can
%% use the \copyrightnoticetext command.  
%\copyrightnoticetext{\copyright IBM, 1990.  Do not open till Xmas.}

% If there is more than one supervisor, use the \supervisor command
% once for each.
\supervisor{Noelle Selin}{Professor, Institute for Data, Systems, and Society and
Department of Earth, Atmospheric and Planetary Sciences\\
Director, Technology and Policy Program}

% This is the department committee chairman, not the thesis committee
% chairman.  You should replace this with your Department's Committee
% Chairman.
\chairman{Noelle Eckley Selin}{Professor, Institute for Data, Systems, and Society and
Department of Earth, Atmospheric and Planetary Sciences\\
Director, Technology and Policy Program}

% Make the titlepage based on the above information.  If you need
% something special and can't use the standard form, you can specify
% the exact text of the titlepage yourself.  Put it in a titlepage
% environment and leave blank lines where you want vertical space.
% The spaces will be adjusted to fill the entire page.  The dotted
% lines for the signatures are made with the \signature command.
\maketitle

% The abstractpage environment sets up everything on the page except
% the text itself.  The title and other header material are put at the
% top of the page, and the supervisors are listed at the bottom.  A
% new page is begun both before and after.  Of course, an abstract may
% be more than one page itself.  If you need more control over the
% format of the page, you can use the abstract environment, which puts
% the word "Abstract" at the beginning and single spaces its text.

%% You can either \input (*not* \include) your abstract file, or you can put
%% the text of the abstract directly between the \begin{abstractpage} and
%% \end{abstractpage} commands.

% First copy: start a new page, and save the page number.
\cleardoublepage
% Uncomment the next line if you do NOT want a page number on your
% abstract and acknowledgments pages.
% \pagestyle{empty}
\setcounter{savepage}{\thepage}
\begin{abstractpage}
% $Log: abstract.tex,v $
% Revision 1.1  93/05/14  14:56:25  start
% Initial revision
% 
% Revision 1.1  90/05/04  10:41:01  levels
% Initial revision
% 
%
%% The text of your abstract and nothing else (other than comments) goes here.
%% It will be single-spaced, and the rest of the text that is supposed to go on
%% the abstract page will be generated by the abstract page environment.  This
%% file should be \input (not \include 'd) from cover.tex.
\begin{flushleft}
    ASGM is the world's largest source of anthropogenic Hg emissions and is common in Latin America, Sub-Saharan Africa, South Asia, and East Asia. However, the amount of mercury emitted from ASGM and contributing to global mercury emissions is subject to substantial uncertainty. Bottom-up studies have quantified sources of Hg, including ASGM, using data on underlying activities to estimate regional and global totals. In contrast, top-down studies have used atmospheric concentration measurements and models to constrain Hg emissions. However, no top-down estimates have yet been calculated for ASGM emissions. With GEOS-Chem's global-scale chemical transport model for Hg, we investigate whether and how ASGM-related Hg emissions can be quantified from existing regional measurement sites for gaseous elemental mercury (GEM). By combining our top-down method with existing bottom-up data, we improve estimates of Hg emissions from ASGM activities, using Peru and the Madre de Dios region of South America as case studies. We find that quantitative constraints on ASGM emissions are better provided by information on the shape of the probability distribution of GEM concentrations, such as the interquartile range and the 95\% range, suggesting possible design guidelines for monitoring networks. The model-based analysis offers insights into improving regional estimates of ASGM emissions. 
\end{flushleft}

\end{abstractpage}

% Additional copy: start a new page, and reset the page number.  This way,
% the second copy of the abstract is not counted as separate pages.
% Uncomment the next 6 lines if you need two copies of the abstract
% page.
% \setcounter{page}{\thesavepage}
% \begin{abstractpage}
% % $Log: abstract.tex,v $
% Revision 1.1  93/05/14  14:56:25  start
% Initial revision
% 
% Revision 1.1  90/05/04  10:41:01  levels
% Initial revision
% 
%
%% The text of your abstract and nothing else (other than comments) goes here.
%% It will be single-spaced, and the rest of the text that is supposed to go on
%% the abstract page will be generated by the abstract page environment.  This
%% file should be \input (not \include 'd) from cover.tex.
\begin{flushleft}
    ASGM is the world's largest source of anthropogenic Hg emissions and is common in Latin America, Sub-Saharan Africa, South Asia, and East Asia. However, the amount of mercury emitted from ASGM and contributing to global mercury emissions is subject to substantial uncertainty. Bottom-up studies have quantified sources of Hg, including ASGM, using data on underlying activities to estimate regional and global totals. In contrast, top-down studies have used atmospheric concentration measurements and models to constrain Hg emissions. However, no top-down estimates have yet been calculated for ASGM emissions. With GEOS-Chem's global-scale chemical transport model for Hg, we investigate whether and how ASGM-related Hg emissions can be quantified from existing regional measurement sites for gaseous elemental mercury (GEM). By combining our top-down method with existing bottom-up data, we improve estimates of Hg emissions from ASGM activities, using Peru and the Madre de Dios region of South America as case studies. We find that quantitative constraints on ASGM emissions are better provided by information on the shape of the probability distribution of GEM concentrations, such as the interquartile range and the 95\% range, suggesting possible design guidelines for monitoring networks. The model-based analysis offers insights into improving regional estimates of ASGM emissions. 
\end{flushleft}

% \end{abstractpage}

\cleardoublepage

\section*{Acknowledgments}
\begin{flushleft}
   I am filled with sincere gratitude as I think about my time at MIT, the journey to get here and all the people that have helped me along the way. Most importantly, I would like to thank Noelle Selin, my research advisor, for welcoming me into her research group.  
   This thesis would not have been possible without the Selin Group's training, challenging questions, and comradery. I am especially grateful to Aryeh Feinberg for his unyielding patience, for helping me with GEOS-Chem, and for encouragement. Thank you to all of the Selin Group's members. I have learned much from you.
       
\end{flushleft}
    
\begin{flushleft}
I would like to thank Barb DeLaBarre, Elena Byrne, Ed Ballo, and Frank Field for everything they do for us. I am grateful to the TPPers in the classes ahead of us who provided invaluable advice on how to navigate MIT since we began our time remotely. Thanks to my TPP cohort (class of 22), we were able to learn, grow, and explore MIT in a supportive environment that we all created and sustained. 
\end{flushleft} 
\begin{flushleft}
I am also grateful to everyone who has contributed to my academic journey. From my primary school teachers in Manzini, Eswatini to all the professors I have had here at MIT, I am grateful. Thank you to my mentors who have provided guidance and support. 
\end{flushleft} 
\begin{flushleft}
    To Moreen: Thank you for your love that has kept me sane and helped me grow. You have been my strength through this journey's ups and downs. I love you.
 \end{flushleft}   
 \begin{flushleft}
    And finally, thank you to my family in Eswatini. I would not be where I am today without your love and support. I love you.
\end{flushleft} 

\begin{flushleft}
    This research was supported by a grant (\#1924148) from the US National Science Foundation.
\end{flushleft} 

%%%%%%%%%%%%%%%%%%%%%%%%%%%%%%%%%%%%%%%%%%%%%%%%%%%%%%%%%%%%%%%%%%%%%%
% -*-latex-*-
